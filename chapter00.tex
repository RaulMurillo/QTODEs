\chapter*{Introduction}
\addcontentsline{toc}{chapter}{Introduction}

The Qualitative Theory of Ordinary Differential Equations takes a rather special place both in Applied Mathematics and in theoretical Mathematics. On the one hand, it is a continuation of the standard lecture on Ordinary Differential Equations (ODEs). On the other hand, it is an introduction to the theory of Dynamical Systems, one of the main mathematical disciplines in recent decades. In addition, it turns out to be very useful for graduates when they meet with differential equations at work, which are usually very complicated and can not be solved by standard methods. If I do not boast too much, the first lecture from Qualitative Theory of ODEs at the Faculty of MIM was made by me in the second half of the 80s of the last century; as you can see, the idea turned out to be successful.

The main idea of a qualitative analysis of differential equations is that without solving the equations themselves, be able to say something about the behavior of solutions.

Therefore, in the first place stand out certain properties such as the stability of solutions. It is stable with respect to changes in the initial conditions of the equation. Note that even with the numerical approach to differential equations all the calculations are subject to certain inevitable error. Therefore, it is good when the asymptotic behavior of the solutions is sensitive to perturbations of the initial state. The first part of the script focuses on this roughly.

Another important concept of this theory is the structural stability. This is the stability of the entire system, i.e. the phase portrait, with regard to perturbation parameters, which usually occur (and in large quantities) on the right side of the equations. In the absence of structural stability, we deal with bifurcations. Methods of qualitative theory allow for precise and accurate enough to examine such bifurcations. We describe them in the third part of the script.

In the case of 2-dimensional autonomous systems, the phase portraits are conceptually quite simple, they consist of singular points, their separatrices and limit cycles;  there are still the peculiarities and blow behavior on the infinite. It is worth mentioning that the problem of limit cycles for polynomial vector fields is the unresolved Hilbert's sixteenth problem. These topics are discussed in the second part of the script.

The fourth part is dedicated to several issues in which there is a small parameter (in a different context). In particular, this class of issues includes the KAM theory and the theory of relaxation oscillations; we discuss them fairly briefly.

In multidimensional systems, new phenomena appear, of which the most important is chaos. The most elementary example of a chaotic system is the famous Smale horseshoe map, defined for a single transformation. In the penultimate part of this script, we will show how Smale horseshoe appears in such elementary systems as a swing moved by periodic external force. We will also give other examples of chaotic behaviors, as attractors.

In the Appendix (Chapter 6), the reader will find the collected main facts from the course lecture on Ordinary Differential Equations.

Each chapter contains a series of tasks (with varying degrees of difficulty) that a self-respecting student should solve.

At the end of the introduction, I would like to thank Professor Zbigniewowi Peradzyńskiemu, who carefully read the manuscript and gave me a list of comments and errors.\\

\begin{flushright}
	Henryk Żoł\c{a}dek,\\
	University of Warsaw,\\
	Faculty of Mathematics, Informatics, and Mechanics, 2011	
\end{flushright}